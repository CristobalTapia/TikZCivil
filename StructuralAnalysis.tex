%Here are defined all kind of drawing, that are useful for the Structural Analysis.
\usetikzlibrary{calc,arrows,shapes,positioning,shadows,trees,patterns,decorations.pathmorphing,decorations.markings}
\tikzset{
    stock/.style={draw=black,text width=10em,line width=0.1em,pattern = north east lines},
    empotrado/.style={draw=none,text width=10em,line width=0.1em,pattern = north east lines},
    columna/.style={color=black, line width=0.1em},
    resorte/.style={color=black, line width=0.1em},
    flecha/.style={thick,<->},
    damper/.style={thick,decoration={markings,  
  		mark connection node=dmp,
  		mark=at position 0.5 with 
  		{
    		\node (dmp) [thick,inner sep=0pt,transform shape,rotate=-90,minimum width=10pt,minimum height=3pt,draw=none] {};
    		\draw [thick] ($(dmp.north east)+(2pt,0)$) -- (dmp.south east) -- (dmp.south west) -- ($(dmp.north west)+(2pt,0)$);
    		\draw [thick] ($(dmp.north)+(0,-3pt)$) -- ($(dmp.north)+(0,3pt)$);
  		}
		}, decorate},
}

%%%%%%%%%%%%%%%%%%%%%%%%%%%%%%%%%%%%%%
%	MarcoDespl:
%%%%%%%%%%%%%%%%%%%%%%%%%%%%%%%%%%%%%
% This command defines a 
\newcommand{\MarcoDespl}[3]{%
    \coordinate (p1) at (#1) {};
    \coordinate (p2) at ([shift={(#3,10em)}]p1) {};
    \coordinate (p3) at ([shift={(10em,0em)}]p1) {};
    \coordinate (p4) at ([shift={({10em+#3},10em)}]p1) {};
    \coordinate (p121) at ([shift={({#3/7},3em)}]p1) {};
    \coordinate (p122) at ([shift={({#3*6/7},7em)}]p1) {};
    \coordinate (p341) at ([shift={({10em+#3/7},3em)}]p1) {};
    \coordinate (p342) at ([shift={({10em+#3*6/7},7em)}]p1) {};
    \draw[columna] plot [smooth, tension=0.8] coordinates{(p1) (p121) (p122) (p2)};
    \draw[columna] plot [smooth, tension=0.8] coordinates{(p3) (p341) (p342) (p4)};
    \coordinate (p5) at ([shift={({10em+#3},9em)}]p1) {};
    %\coordinate[stock] at (5em,10em) (#1) {};
    \filldraw[stock] (p2) rectangle (p5);
    %Definir coordenadas para los apoyos
    \coordinate (p5) at ([shift={(-1.5em,0em)}]p1) {};
    \coordinate (p6) at ([shift={(1.5em,-1em)}]p1) {};
    \coordinate (p7) at ([shift={(8.5em,0em)}]p1) {};
    \coordinate (p8) at ([shift={(11.5em,-1em)}]p1) {};
    \ifnum #2 = 1 {%
	    \filldraw[empotrado] (p5) rectangle (p6);
	    \filldraw[empotrado] (p7) rectangle (p8);};
    \fi

}
%%%%%%%%%%%%%%%%%%%%%%%%%%%%%%%%%%%%%%
%	TMD:
% Tuned Mass Damper
%%%%%%%%%%%%%%%%%%%%%%%%%%%%%%%%%%%%%
% This command defines a Tuned Mass Damper system
\newcommand{\TMD}[2]{%
    \coordinate (p1) at (#1) ;
    \coordinate (p2) at ([shift={(0em,3em)}]p1) {};
    \coordinate (p3) at ([shift={(0em,1.5em)}]p1) {};
    \coordinate (p4) at ([shift={({3.5em+#2},1.5em)}]p1) {};
    %Nodos para la masa
    \coordinate (p5) at ([shift={({3.5em+#2},2.5em)}]p1) {};
    \coordinate (p6) at ([shift={({6.5em+#2},1em)}]p1) {};
    %Nodos para las ruedas
    \coordinate (p7) at ([shift={({4.0em+#2},0.5em)}]p1) {};
    \coordinate (p8) at ([shift={({6.0em+#2},0.5em)}]p1) {};

    \draw[columna, line width=0.3em] (p1) -- (p2);
    \pgfmathsetlengthmacro{\separacion}{(0.35+#2/9)*1em}
    \draw[decoration={aspect=0.3, segment length=\separacion, amplitude=0.4em,coil},decorate] (p3) -- (p4); 
    \filldraw[stock] (p5) rectangle (p6);
    %Dibujar las ruedas
    \draw (p7) circle (0.5em);
    \draw (p8) circle (0.5em);
}
%%%%%%%%%%%%%%%%%%%%%%%%%%%%%%%%%%%%%%
%	TMD:
% Tuned Mass Damper with damping
%%%%%%%%%%%%%%%%%%%%%%%%%%%%%%%%%%%%%
% This command defines a Tuned Mass Damper system
\newcommand{\TMDAmor}[2]{%
    \coordinate (p1) at (#1) ;
    \coordinate (p2) at ([shift={(0em,3em)}]p1) {};
    \coordinate (p3) at ([shift={(0em,2.3em)}]p1) {};
    \coordinate (p4) at ([shift={({3.5em+#2em},2.3em)}]p1) {};
    %Coordenadas para el amortiguamiento
    \coordinate (p9) at ([shift={(0em,1.2em)}]p1) {};
    \coordinate (p10) at ([shift={({3.5em+#2em},1.2em)}]p1) {};
    %Nodos para la masa
    \coordinate (p5) at ([shift={({3.5em+#2em},2.5em)}]p1) {};
    \coordinate (p6) at ([shift={({6.5em+#2em},1em)}]p1) {};
    %Nodos para las ruedas
    \coordinate (p7) at ([shift={({4.0em+#2em},0.5em)}]p1) {};
    \coordinate (p8) at ([shift={({6.0em+#2em},0.5em)}]p1) {};

    \draw[columna, line width=0.3em] (p1) -- (p2);
    \draw[damper, line width=0.1em] (p9) -- (p10);
    \pgfmathsetlengthmacro{\separacion}{(0.35+#2/9)*1em}
    \draw[decoration={aspect=0.3, segment length=\separacion, amplitude=0.4em,coil},decorate] (p3) -- (p4); 
    \filldraw[stock] (p5) rectangle (p6);
    %Dibujar las ruedas
    \draw (p7) circle (0.5em);
    \draw (p8) circle (0.5em);
}


