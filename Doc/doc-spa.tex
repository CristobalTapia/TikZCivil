%        File: doc-spa.tex
%     Created: dom abr 13 12:00  2014 C
% Last Change: dom abr 13 12:00  2014 C
%
\documentclass[11pt,letterpaper,oneside]{book}
\usepackage[left=3cm,right=2cm,top=1cm,bottom=1.5cm,includeheadfoot]{geometry}
\usepackage[spanish]{babel}
%\usepackage[latin1]{inputenc}
\usepackage[utf8x]{inputenc}
\usepackage[T1]{fontenc}
\usepackage{lmodern}
\usepackage{subfig}
\usepackage{listings}
\usepackage{emp}
\usepackage{graphicx}
\usepackage{fancyhdr}
\usepackage{verbatim}   % Comentarios multilinea
\usepackage{tabularx}
\usepackage{booktabs}
\usepackage{ifsym}
\usepackage{float}
\usepackage[font=small]{caption}
\setlength{\parindent}{1cm}
\usepackage{tikz}
%+++++++++++++++++++++++++++++++
% Se importa la libreria
%+++++++++++++++++++++++++++++++
\usepackage{../tikzcivil}

\begin{document}

\begin{titlepage}
  \title{Documentación para el uso de la librería tikzcivil}
  \author{Cristóbal Tapia\\
    \texttt{crtapia@gmail.com}
  }
  \date{\today}
  \maketitle
\end{titlepage}

\lstdefinestyle{customasm}{
  belowcaptionskip=1\baselineskip,
  frame=single,
  xleftmargin=\parindent,
  language=tex,
  basicstyle=\small\ttfamily,
  commentstyle=\itshape\color{green!40!black},
  keywordstyle=\bfseries\color{blue},
  identifierstyle=\color{black},
  stringstyle=\color{purple},
  stepnumber=1,
  tabsize=2,
  breaklines=true,
}
\lstset{style=customasm, numbers=left, texcl=true}


\chapter{Dibujos relacionados con el análisis estructural}
\section{Dinámica}

\subsection{Marco desplazado con un TMD}
\begin{figure}[!htp]
  \centering
  \begin{tikzpicture}[scale=1]
    %Segundo argumento de FrameDispl para empotramientos
    %Tercer argumento de FrameDispl para desplazamiento
    \FrameDispl{0em,0em}{1}{2em}
    % Segundo argumento de TMD es el desplazamiento
    \TMDAmor{4em,10em}{-2}
  \end{tikzpicture}
  \qquad
  \label{fig:rigi2}
\end{figure}

\begin{lstlisting}[firstnumber=1]
\begin{tikzpicture}[scale=1]
  % Segundo argumento de FrameDispl para empotramientos
  % Tercer argumento de FrameDispl para desplazamiento
  \FrameDispl{0em,0em}{1}{2em}
  % Segundo argumento de TMD es el desplazamiento
  \TMDAmor{4em,10em}{-2}
\end{tikzpicture}
\end{lstlisting}

\subsection{Múltiples marcos apilados}

\begin{figure}[!htp]
  \centering
  \subfloat[  ]{%
  \begin{tikzpicture}[scale=0.6]
  % Segundo argumento de FrameDispl para empotramientos
  % Tercer argumento de FrameDispl para desplazamiento
    \FrameDispl{0em,0em}{1}{0em}
    \FrameDispl{0em,10em}{0}{0em}
    \FrameDispl{0em,20em}{0}{0em}
  \end{tikzpicture}
  }
  \qquad
  \subfloat[  ]{%
  \begin{tikzpicture}[scale=0.6]
  % Segundo argumento de FrameDispl para empotramientos
  % Tercer argumento de FrameDispl para desplazamiento
    \FrameDispl{0em,0em}{1}{0.15em}
    \FrameDispl{0.15em,10em}{0}{1.35em}
    \FrameDispl{1.5em,20em}{0}{3.4em}
  \end{tikzpicture}
  }

  \label{fig:rigi1}
\end{figure}

\begin{lstlisting}[firstnumber=1, title=Dibujo izquierdo]
\begin{tikzpicture}[scale=0.6]
  \FrameDispl{0em,0em}{1}{0em}
  \FrameDispl{0em,10em}{0}{0em}
  \FrameDispl{0em,20em}{0}{0em}
\end{tikzpicture}
\end{lstlisting}

\begin{lstlisting}[firstnumber=1, title=Dibujo derecho]
\begin{tikzpicture}[scale=0.6]
  \FrameDispl{0em,0em}{1}{0.15em}
  \FrameDispl{0.15em,10em}{0}{1.35em}
  \FrameDispl{1.5em,20em}{0}{3.4em}
\end{tikzpicture}
\end{lstlisting}


\chapter{Dibujos relacionados con la mecánica de suelos}

\setkeys{fam}{key=default}

\end{document}


