%%%%%%%%%%%%%%%%%%%%%%%%%%%%%%%%%%%%%%%%%%%%%%%%%%%%%%%%%%%%%%%%%%%%%%%%%%%
%* file tikzcivil-geomechanics.tex
%*
%*
%*  Author: Cristóbal Tapia
%*  crtapia at gmail dot com
%*
%*  This program is free software; you can redistribute it and/or modify
%*  it under the terms of the GNU General Public License as published by
%*  the Free Software Foundation; either version 2 of the License, or
%*  (at your option) any later version.
%*
%*  This program is distributed in the hope that it will be useful,
%*  but WITHOUT ANY WARRANTY; without even the implied warranty of
%*  MERCHANTABILITY or FITNESS FOR A PARTICULAR PURPOSE.  See the
%*  GNU General Public License for more details.
%*
%*  You should have received a copy of the GNU General Public License
%*  along with this program; if not, write to the Free Software
%*  Foundation, Inc., 59 Temple Place, Suite 330, Boston, MA  02111-1307 USA
%*
%%%%%%%%%%%%%%%%%%%%%%%%%%%%%%%%%%%%%%%%%%%%%%%%%%%%%%%%%%%%%%%%%%%%%%%%%%%
%% Here are defined all kind of drawing, that are useful for the Geomechanics.
% Author: Cristóbal Tapia Camú. crtapia a gmail dot com
%
%--------------- Definition of command \RetainingWall -------------------------%
%A retaining wall will be defined. 
%------------------------------------------------------------------------------%
%
%Styles for the ground
%This style draws lines on the ground
\newcommand\groundLines[2]{
  \draw[thick] (#1) -- (#2);
  \begin{scope}
    %    \clip (left) rectangle (right);
    \clip (#1) rectangle ([yshift=-0.4cm]#2);
    \foreach \li in {0,0.90,...,5}{
      \draw ([shift={(\li,0)}]#1) -- +(-0.225,-0.225)
      ++(0.15,0) -- +(-0.3,-0.3)
      ++(0.15,0) -- +(-0.3,-0.3)
      ++(-0.075,-0.075) -- +(0.225,-0.225)
      ++(0.075,0.075) -- +(0.3,-0.3)
      ++(0.15,0) -- +(0.3,-0.3);
    }
    %\pgfresetboundingbox
  \end{scope}
}
%This style draws points, which resemble the look of sand
%ToDo: use pgfkeys to pass a value for the thickness of the soil
\newcommand\groundPoints[2]{
  \draw[thick] (#1) -- (#2);
  \begin{scope}
    \clip (#1) rectangle ([yshift=-0.4cm]#2);
    \foreach \li in {0,1.0cm,...,5cm}{
      \draw[very thin] ([shift={(0.1cm+\li,-0.1cm)}]#1) circle (0.03cm)
       ++(0.1cm,-0.05cm) circle (0.01cm)
       ++(0.2cm,-0.01cm) circle (0.02cm)
       ++(0.1cm,0.03cm) circle (0.01cm)
       ++(0.2cm,0.0cm) circle (0.02cm)
       ++(0.1cm,0.05cm) circle (0.01cm);
    }
    %\pgfresetboundingbox
  \end{scope}
}

%Conditionals to know if ground should be drawn
\newif\ifgroundleft
\newif\ifgroundright
%
\pgfkeys{
  /civil/retainingWall/.is family, /civil/retainingWall,
  left ground/.is if = groundleft,
  right ground/.is if = groundright,
  default/.style = {
    position = {0cm,0cm}, base width = 5cm,
    base thickness = 0.8cm, top thickness = 0.5cm,
    bottom thickness = 1cm, height = 5cm,
    wall position ratio = 0.4, fill color = black!20,
    left ground = false, right ground = false,
    left ground height = 2cm, beta = 15,
  },
  position/.estore in = \basePosition,
  base width/.estore in = \baseWidth,
  base thickness/.estore in = \baseThickness,
  top thickness/.estore in = \topThickness,
  bottom thickness/.estore in = \bottomThickness,
  height/.estore in = \height,
  wall position ratio/.estore in = \wallRatio,
  fill color/.estore in = \fillColor,
  left ground height/.estore in = \leftGroundH,
  beta/.estore in = \beta,
}
%
%Definition of the command \retainingWall
\newcommand\RetainingWall[1][]{%
  \pgfkeys{/civil/retainingWall, default, #1,}%
  %Initial position
  \coordinate (w1) at (\basePosition) ;
  %Ground is drown
  \ifgroundleft
    \coordinate (lg1) at ([shift={(-\wallRatio*\baseWidth, \leftGroundH)}]w1);
    \coordinate (lg3) at ([shift={(\wallRatio*\baseWidth, \leftGroundH)}]w1);
    \groundPoints{lg1}{lg3}
  \fi
  %base coords
  \coordinate (b1) at ([yshift=\baseThickness]w1);
  \coordinate (b2) at ([xshift=(\baseWidth*\wallRatio-\bottomThickness/2)]b1);
  \coordinate (b3) at ([shift={(\bottomThickness-\topThickness, \height)}]b2);
  \coordinate (b4) at ([xshift=\topThickness]b3);
  \coordinate (b5) at ([yshift=-\height]b4);
  \coordinate (b6) at ([xshift=\baseWidth*(1-\wallRatio) - \bottomThickness/2]b5);
  \coordinate (b7) at ([yshift=-\baseThickness]b6);
  %Wall is drawn
  \filldraw[style=line wall, fill=\fillColor] (w1) -- (b1) -- (b2) -- (b3) -- (b4) -- (b5)
        -- (b6) -- (b7) -- (w1);
  %ground to the right of the wall (with inclination)
  \ifgroundright
    %Rotation in an angle = beta
    \pgftransformrotate{\beta}
    \coordinate (rg1) at ([xshift=\baseWidth]b4);
    \coordinate (rg2) at ([yshift=-\baseThickness*0.4]rg1);
      \groundPoints{b4}{rg1}
    \pgftransformrotate{-\beta}
    %\filldraw[rotate around={\beta:(b4)}, transform shape, soil style] (b4) rectangle (rg2);
    %\rotatebox{\beta}{
    %\node[soil style,anchor=north, transform shape] at ($(b4)!0.5!(rg1)$){};
    %}
  \fi
} 

\endinput
