%%%%%%%%%%%%%%%%%%%%%%%%%%%%%%%%%%%%%%%%%%%%%%%%%%%%%%%%%%%%%%%%%%%%%%%%%%%
%* file tikzcivil-structural.tex
%*
%*
%*  Author: Cristóbal Tapia
%*  crtapia at gmail dot com
%*
%*  This program is free software; you can redistribute it and/or modify
%*  it under the terms of the GNU General Public License as published by
%*  the Free Software Foundation; either version 2 of the License, or
%*  (at your option) any later version.
%*
%*  This program is distributed in the hope that it will be useful,
%*  but WITHOUT ANY WARRANTY; without even the implied warranty of
%*  MERCHANTABILITY or FITNESS FOR A PARTICULAR PURPOSE.  See the
%*  GNU General Public License for more details.
%*
%*  You should have received a copy of the GNU General Public License
%*  along with this program; if not, write to the Free Software
%*  Foundation, Inc., 59 Temple Place, Suite 330, Boston, MA  02111-1307 USA
%*
%%%%%%%%%%%%%%%%%%%%%%%%%%%%%%%%%%%%%%%%%%%%%%%%%%%%%%%%%%%%%%%%%%%%%%%%%%%
%
%This file contains all the commands related to the structural analysis.
%
%--------------- Definition of command \support -------------------------------%
%Defines the drawing of different kind of supports, capable of rotating
%------------------------------------------------------------------------------%
%Defining conditionals
\newif\iffixed
\newif\ifpinned
\newif\ifsliding
\newif\iffixedsliding
%Defining some auxiliary commands
\newcommand\activatefixed{%
  \fixedtrue
  \pinnedfalse
  \slidingfalse
  \fixedslidingfalse
}
%
\newcommand\activatepinned{%
  \fixedfalse
  \pinnedtrue
  \slidingfalse
  \fixedslidingfalse
}
%
\newcommand\activatesliding{%
  \fixedfalse
  \pinnedfalse
  \slidingtrue
  \fixedslidingfalse
}
%
\newcommand\activatefixedsliding{%
  \fixedfalse
  \pinnedfalse
  \slidingfalse
  \fixedslidingtrue
}
%
%Style for the ground
\tikzset{
  soil style/.style={draw=none,text width=10em,line width=0.1em,pattern = north east lines},
}
\pgfkeys{
  /civil/supports/.is family, /civil/supports,
  default/.style = {
    position = {0cm,0cm}, angle = 0,
    type = fixed, width = 1cm,
  },
  type/.is choice,
  type/fixed/.code = {\activatefixed},
  type/pinned/.code = {\activatepinned},
  type/sliding/.code = {\activatesliding},
  type/fixedsliding/.code = {\activatefixedsliding},
  position/.estore in = \supportPosition,
  angle/.estore in = \supportAngle,
  width/.estore in = \supportWidth,
}
%
%Definition of the command begins
\newcommand\Support[1][]{%
  \pgfkeys{/civil/supports, default, #1,}%
  %Initial position
  \coordinate (k1) at (\supportPosition) ;
  %If the type of the support is fixed
  \iffixed
    %Rotate if requested
    \pgftransformrotate{\supportAngle}
    \coordinate (k2) at ([shift={(-\supportWidth/2,0cm)}]k1) {};
    \coordinate (k3) at ([shift={(\supportWidth/2,-\supportWidth/3)}]k1) {};
    \filldraw[soil style] (k2) rectangle (k3);
    \pgftransformrotate{-\supportAngle}
  \fi
  %If the type of the support is pinned
  \ifpinned
    %Rotate if requested
    \pgftransformrotate{\supportAngle}
    %Coordinates for the trianlge
    \coordinate (k4) at ([shift={(-\supportWidth/2,-\supportWidth*2/3)}]k1) {};
    \coordinate (k5) at ([shift={(\supportWidth/2,-\supportWidth*2/3)}]k1) {};
    %Triangle is drawn
    \draw (k1) -- (k4) -- (k5) -- (k1);
    %This are te coordinates for the lower-right point of the soil-rectangle
    \coordinate (k6) at ([shift={(\supportWidth/2,-\supportWidth)}]k1) {};
    %The rectangle for the soil is drawn
    \filldraw[soil style] (k4) rectangle (k6);
    \pgftransformrotate{-\supportAngle}
  \fi
  %If the type of the support is pinned and sliding
  \ifsliding
    %Rotate if requested
    \pgftransformrotate{\supportAngle}
    %Coordinates for the triangle
    \coordinate (k4) at ([shift={(-\supportWidth/2,-\supportWidth*2/3)}]k1) {};
    \coordinate (k5) at ([shift={(\supportWidth/2,-\supportWidth*2/3)}]k1) {};
    %Trianlge is drawn
    \draw (k1) -- (k4) -- (k5) -- (k1);
    %Wheels are drawn
    \coordinate (k6) at ([shift={(0,-\supportWidth*2/3-\supportWidth/10)}]k1);
    \coordinate (k7) at ([shift={(-\supportWidth*2/5,-\supportWidth*2/3-\supportWidth/10)}]k1);
    \coordinate (k8) at ([shift={(\supportWidth*2/5,-\supportWidth*2/3-\supportWidth/10)}]k1);
    \foreach \center in {{(k6)}, {(k7)}, {(k8)}} 
      \draw \center circle (\supportWidth/10);
    %Coordinates for the soil under the wheels
    \coordinate (k9) at ([shift={(0,-\supportWidth/5)}]k4);
    \coordinate (k10) at ([shift={(0,-\supportWidth/5-\supportWidth/3)}]k5);
    \filldraw[soil style = soil1] (k9) rectangle (k10);
    \pgftransformrotate{-\supportAngle}
  \fi
  %If the type is fixed-sliding
  \iffixedsliding
    %Rotate if requested
    \pgftransformrotate{\supportAngle}
    %Line
    \coordinate (k6) at ([shift={(-\supportWidth/2,-\supportWidth/5)}]k1);
    \coordinate (k7) at ([shift={(\supportWidth/2,-\supportWidth/5)}]k1);
    \coordinate (k13) at ([shift={(0,-\supportWidth*2/10)}]k1);
    \draw[line width = 2pt] (k6) -- (k7);
    \draw[line width = 2pt] (k1) -- (k13);
    %Wheels are drawn
    \coordinate (k8) at ([shift={(0,-\supportWidth*3/10)}]k1);
    \coordinate (k9) at ([shift={(-\supportWidth*2/5,-\supportWidth*3/10)}]k1);
    \coordinate (k10) at ([shift={(\supportWidth*2/5,-\supportWidth*3/10)}]k1);
    \foreach \center in {{(k8)}, {(k9)}, {(k10)}}
      \draw \center circle (\supportWidth/10);
    %Coordinates for the soil under the wheels
    \coordinate (k11) at ([shift={(0,-\supportWidth/5)}]k6);
    \coordinate (k12) at ([shift={(0,-\supportWidth/5-\supportWidth/3)}]k7);
    \filldraw[soil style = soil1] (k11) rectangle (k12);
    \pgftransformrotate{-\supportAngle}
  \fi
}
%End of the definition
%
%----------------- Definition of command \MassWithSpring: ---------------------%
%Defines the drawing of a typical mass tied to a spring. Optionally it can also 
%have a damper.
%------------------------------------------------------------------------------%
%
%pgfkeys are defined first
%Definition of 'if-keys'
\newif\ifdamper
\newif\ifwall
%
\pgfkeys{
  /civil/massWithSpring/.is family, /civil/massWithSpring,
  with damper/.is if = damper,
  with wall/.is if = wall,
  defaultMass/.style = {
    position = {0em,0em}, displacement = 0em,
    with damper = false, with wall = true,
  },
  position/.estore in = \massPosition,
  displacement/.estore in = \massDisplacement,
}
%
%Definition of the command to draw
\newcommand\MassWithSpring[1][]{%
  \pgfkeys{/civil/massWithSpring, defaultMass, #1}%
  \coordinate (p1) at (\massPosition) ;
  \coordinate (p2) at ([shift={(0em,4em)}]p1) {};
  %The position of the spring varies with the presence or
  %absence of the damper
  \ifdamper
    \coordinate (p3) at ([shift={(0em,3.0em)}]p1) {};
    \coordinate (p4) at ([shift={({4.5em+\massDisplacement},3.0em)}]p1) {};
  \else
    \coordinate (p3) at ([shift={(0em,2.2em)}]p1) {};
    \coordinate (p4) at ([shift={({4.5em+\massDisplacement},2.2em)}]p1) {};
  \fi
  %Coordinates for the mass
  \coordinate (p5) at ([shift={({4.5em+\massDisplacement},3.5em)}]p1) {};
  \coordinate (p6) at ([shift={({8.5em+\massDisplacement},1.0em)}]p1) {};
  %Coordinates for the wheels
  \coordinate (p7) at ([shift={({5.3em+\massDisplacement},0.5em)}]p1) {};
  \coordinate (p8) at ([shift={({7.7em+\massDisplacement},0.5em)}]p1) {};
  %The wall will be drawn if requested
  \ifwall
    %Coordinates for the damper
    \coordinate (p9) at ([shift={(0em,1.6em)}]p1) {};
    \coordinate (p10) at ([shift={({4.5em+\massDisplacement},1.6em)}]p1) {};
    \draw[columna, line width=0.3em] (p1) -- (p2);
  \fi
  %Damper is applied if requested
  \ifdamper
    \draw[damper, line width=0.1em] (p9) -- (p10);
  \fi
  \pgfmathsetlengthmacro{\separation}{(4.5+\massDisplacement/9)}
  \draw[decoration={aspect=0.3, segment length=\separation, amplitude=0.5em,coil},decorate] (p3) -- (p4);
  \filldraw[stock] (p5) rectangle (p6);
  %Wheels are drawn
  \draw (p7) circle (0.5em);
  \draw (p8) circle (0.5em);
}
%
%--------------------- Frame --------------------------------------------------%
%This command defines a frame with a mass
%------------------------------------------------------------------------------%
% Definition of if-keys
\newif\ifsupport
%
\pgfkeys{
  /civil/frame/.is family, /civil/frame,
  with damper/.is if = damper,
  with support/.is if = support,
  defaultFrame/.style = {
    height = 4cm, width = 4cm,
    position = {0em,0em}, displacement = 0em,
    with damper = false, with support = true,
    mass thickness = 0.5cm,
  },
  height/.estore in = \frameHeight,%Story height
  width/.estore in = \frameWidth,%frame width
  mass thickness/.estore in = \massThickness,%thickness of the floor
  position/.estore in = \framePosition,%position of the left support
  displacement/.estore in = \frameDisplacement,%horizontal displacement at the top
}
%
\newcommand{\Frame}[1][]{%
  \pgfkeys{/civil/frame, defaultFrame, #1}%
  %\pgfkeys{/civil/supports, defaultFrame, #1}%
  \coordinate (p1) at (\framePosition) {};
  \coordinate (p2) at ([shift={(\frameDisplacement,\frameHeight)}]p1) {};
  \coordinate (p3) at ([shift={(\frameWidth,0)}]p1) {};
  \coordinate (p4) at ([shift={({\frameWidth+\frameDisplacement},\frameHeight)}]p1) {};
  \coordinate (p121) at ([shift={({\frameDisplacement/7},\frameHeight*1/3)}]p1) {};
  \coordinate (p122) at ([shift={({\frameDisplacement*6/7},\frameHeight*2/3)}]p1) {};
  \coordinate (p341) at ([shift={({\frameWidth+\frameDisplacement/7},\frameHeight*1/3)}]p1) {};
  \coordinate (p342) at ([shift={({\frameWidth+\frameDisplacement*6/7},\frameHeight*2/3)}]p1) {};
  %Columns are drawn
  \draw[columna] plot [smooth, tension=0.8] coordinates{(p1) (p121) (p122) (p2)};
  \draw[columna] plot [smooth, tension=0.8] coordinates{(p3) (p341) (p342) (p4)};
  %Mass-plate is drawn
  \coordinate (p5) at ([shift={({\frameWidth+\frameDisplacement},\frameHeight-\massThickness)}]p1) {};
  \filldraw[stock] (p2) rectangle (p5);
  %Supports are drawn
  \ifsupport
    \Support[position = p1];
    \Support[position = p3];
  \fi
  %Damper is added if specified
  \ifdamper
    %Define coordinates for the damper
    \coordinate (p6) at ([xshift=-(\frameWidth+\frameDisplacement)/6]$(p1)!0.5!(p5)$);
    \coordinate (p7) at ([xshift=(\frameWidth+\frameDisplacement)/6]$(p1)!0.5!(p5)$);
    %Draw the damper
    \draw[line width=0.1em] (p1) -- (p6);
    \draw[line width=0.1em] (p7) -- (p5);
    \draw[damper, line width=0.1em] (p6) -- (p7);
  \fi
}
%
%------------------- Definition of command \FrameSimple -----------------------%
%Defines a simple frame, where you can make each dof move independently.
%------------------------------------------------------------------------------%
%
\pgfkeys{
  /civil/frameSimple/.is family, /civil/frameSimple,
  with damper/.is if = damper,
  with support/.is if = support,
  defaultFrame/.style = {
    height = 4cm, width = 4cm,
    position = {0em,0em}, displacement = 0em,
    with damper = false, with support = true,
    mass thickness = 0.5cm,
  },
  height/.estore in = \frameHeight,%Story height
  width/.estore in = \frameWidth,%frame width
  mass thickness/.estore in = \massThickness,%thickness of the floor
  position/.estore in = \framePosition,%position of the left support
  displacement/.estore in = \frameDisplacement,%horizontal displacement at the top
}
%
%Definition of the command
\newcommand{\FrameSimple}[1][]{%
  \pgfkeys{/civil/frameSimple, defaultFrame, #1}%
  \coordinate (p1) at (\framePosition) {};
  \coordinate (p2) at ([shift={(\frameDisplacement,\frameHeight)}]p1) {};
  \coordinate (p3) at ([shift={(\frameWidth,0)}]p1) {};
  \coordinate (p4) at ([shift={({\frameWidth+\frameDisplacement},\frameHeight)}]p1) {};
  \coordinate (p121) at ([shift={({\frameDisplacement/7},\frameHeight*1/3)}]p1) {};
  \coordinate (p122) at ([shift={({\frameDisplacement*6/7},\frameHeight*2/3)}]p1) {};
  \coordinate (p341) at ([shift={({\frameWidth+\frameDisplacement/7},\frameHeight*1/3)}]p1) {};
  \coordinate (p342) at ([shift={({\frameWidth+\frameDisplacement*6/7},\frameHeight*2/3)}]p1) {};
  %Columns are drawn
  \draw[columna] plot [smooth, tension=0.8] coordinates{(p1) (p121) (p122) (p2)};
  \draw[columna] plot [smooth, tension=0.8] coordinates{(p3) (p341) (p342) (p4)};
  %Beam on the top is drawn
  \draw[columna] plot [smooth, tension=0.8] coordinates{(p2) (p4)};
  %Supports are drawn
  \ifsupport
    \Support[position = p1];
    \Support[position = p3];
  \fi
%
}
\endinput
