%%%%%%%%%%%%%%%%%%%%%%%%%%%%%%%%%%%%%%%%%%%%%%%%%%%%%%%%%%%%%%%%%%%%%%%%%%%
%* file tikzcivil-structural.tex
%*
%*
%*  Author: Cristóbal Tapia
%*  crtapia at gmail dot com
%*
%*  This program is free software; you can redistribute it and/or modify
%*  it under the terms of the GNU General Public License as published by
%*  the Free Software Foundation; either version 2 of the License, or
%*  (at your option) any later version.
%*
%*  This program is distributed in the hope that it will be useful,
%*  but WITHOUT ANY WARRANTY; without even the implied warranty of
%*  MERCHANTABILITY or FITNESS FOR A PARTICULAR PURPOSE.  See the
%*  GNU General Public License for more details.
%*
%*  You should have received a copy of the GNU General Public License
%*  along with this program; if not, write to the Free Software
%*  Foundation, Inc., 59 Temple Place, Suite 330, Boston, MA  02111-1307 USA
%*
%%%%%%%%%%%%%%%%%%%%%%%%%%%%%%%%%%%%%%%%%%%%%%%%%%%%%%%%%%%%%%%%%%%%%%%%%%%
%
%This file contains all the commands related to the structural analysis.
%
%--------------- Definition of command \axisTwo -------------------------------%
%Defines an axis with two perpendicular directions. This is
%intended to be used in the description of degrees of freedom or to define
%the coordinate system of a drawing.
%------------------------------------------------------------------------------%
%
\pgfkeys{
  /civil/axistwo/.is family, /civil/axistwo,
  default/.style = {
    position = {0cm,0cm},
    rotate   = 0,
    size     = 30pt,
  },
  position/.store in = \axisPosition,
  rotate/.store in   = \axisAngle,
  size/.store in     = \axisSize,
}

\newcommand\axisTwo[3][]{%
  \pgfkeys{/civil/axistwo, default, #1,}%
  \coordinate (p0) at (\axisPosition) {};
  \node (px) at ([xshift=\axisSize]p0) {#2};
  \node (py) at ([yshift=\axisSize]p0) {#3};
  %Axis will be drawn
  \draw [->, thick] (p0) -- (px);
  \draw [->, thick] (p0) -- (py);
}


%--------------- Definition of command \axisTwoRot ----------------------------%
%Similar to de \axisTwo command, but with a rotation degree of freedom.
%------------------------------------------------------------------------------%
%
\pgfkeys{
  /civil/axistworot/.is family, /civil/axistworot,
  default/.style = {
    position = {0cm,0cm}, rotate = 0,
    size = 30pt,
  },
  position/.store in = \axisPosition,
  rotate/.store in = \axisAngle,
  size/.store in = \axisSize,
}
\newcommand\axisTwoRot[4][]{%
  \pgfkeys{/civil/axistworot, default, #1,}%
  \coordinate (p0) at (\axisPosition) {};
  \node (px) at ([xshift=\axisSize]p0) {#2};
  \node (py) at ([yshift=\axisSize]p0) {#3};
  \coordinate (pr) at ([xshift=\axisSize*1/2*cos(-20)]p0) {};
  \coordinate (pr) at ([yshift=\axisSize*1/2*sin(-20)]pr) {};
  \coordinate (prname) at ([xshift=\axisSize*4/5*cos(120)]p0) {};
  \node (prname) at ([yshift=\axisSize*4/5*sin(120)]prname) {#4};
  %Axis will be drawn
  \draw [->, thick] (p0) -- (px);
  \draw [->, thick] (p0) -- (py);
  \draw [->, thick] (pr) arc (-20:160:\axisSize*1/2);
}

%--------------- Definition of command \support -------------------------------%
%Defines the drawing of different kind of supports, capable of rotating
%------------------------------------------------------------------------------%
%Defining conditionals
\newif\iffixed
\newif\ifpinned
\newif\ifsliding
\newif\iffixedsliding
%Defining some auxiliary commands
\newcommand\activatefixed{%
  \fixedtrue
  \pinnedfalse
  \slidingfalse
  \fixedslidingfalse
}
%
\newcommand\activatepinned{%
  \fixedfalse
  \pinnedtrue
  \slidingfalse
  \fixedslidingfalse
}
%
\newcommand\activatesliding{%
  \fixedfalse
  \pinnedfalse
  \slidingtrue
  \fixedslidingfalse
}
%
\newcommand\activatefixedsliding{%
  \fixedfalse
  \pinnedfalse
  \slidingfalse
  \fixedslidingtrue
}
%
\pgfkeys{
  /civil/supports/.is family, /civil/supports,
  default/.style = {
    position = {0cm,0cm},
    angle    = 0,
    type     = fixed,
    width    = 1cm,
  },
  type/.is choice,
  type/fixed/.code        = {\activatefixed},
  type/pinned/.code       = {\activatepinned},
  type/sliding/.code      = {\activatesliding},
  type/fixedsliding/.code = {\activatefixedsliding},
  position/.store in      = \supportPosition,
  angle/.store in         = \supportAngle,
  width/.store in         = \supportWidth,
}
%--------Definition of the command begins-------------------------------------%
\newcommand\Support[1][]{%
  \pgfkeys{/civil/supports, default, #1,}%
  %Initial position
  \coordinate (k1) at (\supportPosition) ;
  %If the type of the support is fixed
  \iffixed
    %Rotate if requested
    \pgftransformrotate{\supportAngle}
    \coordinate (k2) at ([shift={(-\supportWidth/2,0cm)}]k1) {};
    \coordinate (k3) at ([shift={(\supportWidth/2,-\supportWidth/3)}]k1) {};
    \filldraw[soil style] (k2) rectangle (k3);
    \pgftransformrotate{-\supportAngle}
  \fi
  %If the type of the support is pinned
  \ifpinned
    %Rotate if requested
    \pgftransformrotate{\supportAngle}
    %Coordinates for the trianlge
    \coordinate (k4) at ([shift={(-\supportWidth*1/3,-\supportWidth*1/3)}]k1) {};
    \coordinate (k5) at ([shift={(\supportWidth*1/3,-\supportWidth*1/3)}]k1) {};
    %Triangle is drawn
    \draw (k1) -- (k4) -- (k5) -- (k1);
    %This are te coordinates for the lower-right point of the soil-rectangle
    \coordinate (k6) at ([shift={(\supportWidth/2,-\supportWidth*2/3)}]k1) {};
    %This are te coordinates for the upper-left point of the soil-rectangle
    \coordinate (k7) at ([shift={(-\supportWidth/2,-\supportWidth*1/3)}]k1) {};
    %The rectangle for the soil is drawn
    \filldraw[soil style] (k7) rectangle (k6);
    \pgftransformrotate{-\supportAngle}
  \fi
  %If the type of the support is pinned and sliding
  \ifsliding
    %Rotate if requested
    \pgftransformrotate{\supportAngle}
    %Coordinates for the triangle
    \coordinate (k4) at ([shift={(-\supportWidth*1/3,-\supportWidth*1/3)}]k1) {};
    \coordinate (k5) at ([shift={(\supportWidth*1/3,-\supportWidth*1/3)}]k1) {};
    %Trianlge is drawn
    \draw (k1) -- (k4) -- (k5) -- (k1);
    %Wheels are drawn
    \coordinate (k6) at ([shift={(0,-\supportWidth*1/3-\supportWidth/10)}]k1);
    \coordinate (k7) at ([shift={(-\supportWidth*1/5,-\supportWidth*1/3-\supportWidth/10)}]k1);
    \coordinate (k8) at ([shift={(\supportWidth*1/5,-\supportWidth*1/3-\supportWidth/10)}]k1);
    \foreach \center in {{(k6)}, {(k7)}, {(k8)}}
      \draw \center circle (\supportWidth/10);
    %Coordinates for the soil under the wheels
    \coordinate (k9) at ([shift={(-\supportWidth/2,-\supportWidth/10)}]k6);
    \coordinate (k10) at ([shift={(\supportWidth/2,-\supportWidth/3-\supportWidth/10)}]k6);
    \filldraw[soil style] (k9) rectangle (k10);
    \pgftransformrotate{-\supportAngle}
  \fi
  %If the type is fixed-sliding
  \iffixedsliding
    %Rotate if requested
    \pgftransformrotate{\supportAngle}
    %Line
    \coordinate (k6) at ([shift={(-\supportWidth/2,-\supportWidth/5)}]k1);
    \coordinate (k7) at ([shift={(\supportWidth/2,-\supportWidth/5)}]k1);
    \coordinate (k13) at ([shift={(0,-\supportWidth*2/10)}]k1);
    \draw[line width = 2pt] (k6) -- (k7);
    \draw[line width = 2pt] (k1) -- (k13);
    %Wheels are drawn
    \coordinate (k8) at ([shift={(0,-\supportWidth*3/10)}]k1);
    \coordinate (k9) at ([shift={(-\supportWidth*2/5,-\supportWidth*3/10)}]k1);
    \coordinate (k10) at ([shift={(\supportWidth*2/5,-\supportWidth*3/10)}]k1);
    \foreach \center in {{(k8)}, {(k9)}, {(k10)}}
      \draw \center circle (\supportWidth/10);
    %Coordinates for the soil under the wheels
    \coordinate (k11) at ([shift={(0,-\supportWidth/5)}]k6);
    \coordinate (k12) at ([shift={(0,-\supportWidth/5-\supportWidth/3)}]k7);
    \filldraw[soil style] (k11) rectangle (k12);
    \pgftransformrotate{-\supportAngle}
  \fi
}
%End of the definition
%
%----------------- Definition of command \MassWithSpring: ---------------------%
%Defines the drawing of a typical mass tied to a spring. Optionally it can also
%have a damper.
%------------------------------------------------------------------------------%
%
%pgfkeys are defined first
%Definition of 'if-keys'
\newif\ifdamper
\newif\ifwall
%
\pgfkeys{
  /civil/massWithSpring/.is family, /civil/massWithSpring,
  draw damper/.is if = damper,
  with wall/.is if   = wall,
  defaultMass/.style = {
    position     = {0em,0em},
    displacement = 0em,
    with wall    = true,
    draw damper  = false,
  },
  with damper/.style = {
    draw damper = true,
  },
  position/.store in     = \massPosition,
  displacement/.store in = \massDisplacement,
}
%
%Definition of the command to draw
\newcommand\MassWithSpring[1][]{%
  \pgfkeys{/civil/massWithSpring, defaultMass, #1}%
  \coordinate (p1) at (\massPosition) ;
  \coordinate (p2) at ([shift={(0em,4em)}]p1) {};
  %The position of the spring varies with the presence or
  %absence of the damper
  \ifdamper
    \coordinate (p3) at ([shift={(0em,3.0em)}]p1) {};
    \coordinate (p4) at ([shift={({4.5em+\massDisplacement},3.0em)}]p1) {};
  \else
    \coordinate (p3) at ([shift={(0em,2.2em)}]p1) {};
    \coordinate (p4) at ([shift={({4.5em+\massDisplacement},2.2em)}]p1) {};
  \fi
  %Coordinates for the mass
  \coordinate (p5) at ([shift={({4.5em+\massDisplacement},3.5em)}]p1) {};
  \coordinate (p6) at ([shift={({8.5em+\massDisplacement},1.0em)}]p1) {};
  %Coordinates for the wheels
  \coordinate (p7) at ([shift={({5.3em+\massDisplacement},0.5em)}]p1) {};
  \coordinate (p8) at ([shift={({7.7em+\massDisplacement},0.5em)}]p1) {};
  %The wall will be drawn if requested
  \ifwall
    %Coordinates for the damper
    \coordinate (p9) at ([shift={(0em,1.6em)}]p1) {};
    \coordinate (p10) at ([shift={({4.5em+\massDisplacement},1.6em)}]p1) {};
    \draw[column, line width=0.3em] (p1) -- (p2);
  \fi
  %Damper is applied if requested
  \ifdamper
    \draw[damper] (p9) -- (p10);
  \fi
  \pgfmathsetlengthmacro{\separation}{(4.5+\massDisplacement/9)}
  \draw[spring] (p3) -- (p4);
  \filldraw[ground] (p5) rectangle (p6);
  %Wheels are drawn
  \draw (p7) circle (0.5em);
  \draw (p8) circle (0.5em);
}
%
%--------------------- Frame --------------------------------------------------%
%This command defines a frame with a mass
%------------------------------------------------------------------------------%
% Definition of if-keys
\newif\ifsupport
%
\pgfkeys{
  /civil/frame/.is family, /civil/frame,
  with damper/.is if  = damper,
  with support/.is if = support,
  defaultFrame/.style = {
    height         = 4cm,
    width          = 4cm,
    position       = {0em,0em},
    displacement   = 0em,
    with damper    = false,
    with support   = true,
    mass thickness = 0.5cm,
  },
  height/.store in         = \frameHeight,%Story height
  width/.store in          = \frameWidth,%frame width
  mass thickness/.store in = \massThickness,%thickness of the floor
  position/.store in       = \framePosition,%position of the left support
  displacement/.store in   = \frameDisplacement,%horizontal displacement at the top
}
%
\newcommand{\Frame}[1][]{%
  \pgfkeys{/civil/frame, defaultFrame, #1}%
  %\pgfkeys{/civil/supports, defaultFrame, #1}%
  \coordinate (p1) at (\framePosition) {};
  \coordinate (p2) at ([shift={(\frameDisplacement,\frameHeight)}]p1) {};
  \coordinate (p3) at ([shift={(\frameWidth,0)}]p1) {};
  \coordinate (p4) at ([shift={({\frameWidth+\frameDisplacement},\frameHeight)}]p1) {};
  %
  \coordinate (p121) at ([shift={(0,\massThickness)}]p1) {};
  \coordinate (p122) at ([shift={(0,-\massThickness)}]p2) {};
  \coordinate (p341) at ([shift={(0,\massThickness)}]p3) {};
  \coordinate (p342) at ([shift={(0,-\massThickness)}]p4) {};
  %Columns are drawn
  \draw[column] (p1) to[out=90,in=-90] (p121) to[out=90,in=-90] (p122) to[out=90,in=-90] (p2);
  \draw[column] (p3) to[out=90,in=-90] (p341) to[out=90,in=-90] (p342) to[out=90,in=-90] (p4);
  %Mass-plate is drawn
  \coordinate (p5) at ([shift={({\frameWidth+\frameDisplacement},\frameHeight-\massThickness)}]p1) {};
  \filldraw[ground] (p2) rectangle (p5);
  %Supports are drawn
  \ifsupport
    \Support[position = p1];
    \Support[position = p3];
  \fi
  %Damper is added if specified
  \ifdamper
    %Define coordinates for the damper
    \coordinate (p6) at ([xshift=-(\frameWidth+\frameDisplacement)/6]$(p1)!0.5!(p5)$);
    \coordinate (p7) at ([xshift=(\frameWidth+\frameDisplacement)/6]$(p1)!0.5!(p5)$);
    %Draw the damper
    \draw[line width=0.1em] (p1) -- (p6);
    \draw[line width=0.1em] (p7) -- (p5);
    \draw[damper, line width=0.1em] (p6) -- (p7);
  \fi
}
%
%------------------- Definition of command \FrameSimple -----------------------%
%Defines a simple frame, where you can make each dof move independently.
%------------------------------------------------------------------------------%
%This frame has six degree of freedom (three at each of the upper nodes).
%Each of them can be moved independently.
\pgfkeys{
  /civil/frameSimple/.is family, /civil/frameSimple,
  with damper/.is if  = damper,
  with support/.is if = support,
  defaultFrame/.style = {
    height         = 4cm,
    width          = 4cm,
    position       = {0em,0em},
    displacement   = 0em,
    with damper    = false,
    with support   = true,
    mass thickness = 0.5cm,
    left support   = fixed,
    right support  = fixed,
    %Default values for each dof
    dof1=0, dof2=0, dof3=0,
    dof4=0, dof5=0, dof6=0,
    dof7=0, dof8=0, dof9=0,
    dof10=0, dof11=0, dof12=0,
  },
  height/.store in         = \frameHeight,%Story height
  width/.store in          = \frameWidth,%frame width
  mass thickness/.store in = \massThickness,%thickness of the floor
  position/.store in       = \framePosition,%position of the left support
  displacement/.store in   = \frameDisplacement,%horizontal displacement at the top
  left support/.store in   = \leftSupport,
  right support/.store in  = \rightSupport,
  dof1/.store in  = \dofOne,
  dof2/.store in  = \dofTwo,
  dof3/.store in  = \dofThree,
  dof4/.store in  = \dofFour,
  dof5/.store in  = \dofFive,
  dof6/.store in  = \dofSix,
  dof7/.store in  = \dofSeven,
  dof8/.store in  = \dofEight,
  dof9/.store in  = \dofNine,
  dof10/.store in = \dofTen,
  dof11/.store in = \dofEleven,
  dof12/.store in = \dofTwelve,
}
%
%Definition of the command
\newcommand{\FrameSimple}[1][]{%
  \pgfkeys{/civil/frameSimple, defaultFrame, #1}%
  %Definition of coordinates of the nodes
  \coordinate (p0) at (\framePosition) {};
  \coordinate (p1) at ([shift={(\dofSeven,\dofEight)}]p0) {};
  \coordinate (p2) at ([shift={(\dofOne,\frameHeight+\dofTwo)}]p0) {};
  \coordinate (p3) at ([shift={(\frameWidth+\dofFour,\frameHeight+\dofFive)}]p0) {};
  \coordinate (p4) at ([shift={(\frameWidth+\dofTen,\dofEleven)}]p0) {};
  %The following coordinates are used when drawing the elements to
  %make the curves look better
  \pgftransformrotate{\dofNine}
  \coordinate (p11) at ([shift={(0,\frameHeight/10)}]p1);
  \pgftransformrotate{-\dofNine}
  \pgftransformrotate{\dofThree}
  \coordinate (p22) at ([shift={(\frameWidth/10,0)}]p2);
  \coordinate (p21) at ([shift={(0,-\frameHeight/10)}]p2);
  \pgftransformrotate{-\dofThree}
  \pgftransformrotate{\dofSix}
  \coordinate (p32) at ([shift={(-\frameWidth/10,0)}]p3);
  \coordinate (p33) at ([shift={(0,-\frameHeight/10)}]p3);
  \pgftransformrotate{-\dofSix}
  \pgftransformrotate{\dofTwelve}
  \coordinate (p43) at ([shift={(0,\frameHeight/10)}]p4);
  \pgftransformrotate{-\dofTwelve}
  %Frame element 1
  \draw[frame element] (p1) to[out=90+\dofNine,in=-90+\dofNine]
                       (p11) to[out=90+\dofNine,in=-90+\dofThree]
                       (p21) to[out=90+\dofThree,in=-90+\dofThree] (p2);
  %Frame element 2
  \draw[frame element] (p2) to[out=0+\dofThree,in=180+\dofThree]
                       (p22) to[out=0+\dofThree,in=180+\dofSix]
                       (p32) to[out=0+\dofSix,in=180+\dofSix] (p3);
  %Frame element 3
  \draw[frame element] (p3) to[out=-90+\dofSix,in=90+\dofSix]
                       (p33) to[out=-90+\dofSix,in=90+\dofTwelve]
                       (p43) to[out=-90+\dofTwelve,in=90+\dofTwelve] (p4);
  %Supports are drawn
  \ifsupport
    \Support[position = p1, type=\leftSupport];
    \Support[position = p4, type=\rightSupport];
  \fi
%
}

%%%%%%%%%%%%%%%%%%%%%%%%%%%%%%%%%%%%%%%%%%%%%%%%%%%%%%%%%%%%%%%%%%%%%%%%%%%%%%%%
% Timber constructions
%%%%%%%%%%%%%%%%%%%%%%%%%%%%%%%%%%%%%%%%%%%%%%%%%%%%%%%%%%%%%%%%%%%%%%%%%%%%%%%%
% The following drowings are related to timber structures.
%
%%%%%%%%%%%%%%%%%%%%%%%%%%%%%%%%%%%%%%%%%%%%%%%%%%%%%%%%%%%%%%%%%%%%%%%%%%%%%%%%
% Beams with holes
%%%%%%%%%%%%%%%%%%%%%%%%%%%%%%%%%%%%%%%%%%%%%%%%%%%%%%%%%%%%%%%%%%%%%%%%%%%%%%%%
% Beams with holes, as described in the german code DIN 1052
%
%Conditionals are defined
\newif\ifleftsupport
\newif\ifrightsupprt
\newif\ifhole
%pgfkeys are defined
\pgfkeys{
  /civil/timberBeam/.is family, /civil/timberBeam,
  with support/.is if       = support,
  with hole/.is if          = hole,
  draw left supprt/.is if   = leftsupport,
  draw right support/.is if = rightsupport,
  defaultTimberBeam/.style = {
    height        = 4cm,
    width         = 8cm,
    position      = {0cm,0cm},
    with support  = true,
    left support  = pinned,
    right support = pinned,
    hole position = center,
    with hole     = false,
  },
  with left support/.style = {
    draw left support = true,
  },
  with right supprt/.style = {
    draw right support = true,
  },
  hole position/.is choice,
  hole position/center/.style = {
    hole position x = \beamWidth/2,
    hole position y = \beamHeight/2,
  },
  hole position/right/.style = {
    hole position x = \beamHeight - \holeDiameter,
    hole position y = \beamHeight/2,
  },
  hole position x/.store in = \holePosX,
  hole position y/.store in = \holePosY,
  height/.store in          = \beamHeight,%Story height
  width/.store in           = \beamWidth,%frame width
  position/.store in        = \beamPosition,%position of the left support
  left support/.store in    = \leftSupport,  %type of support
  right support/.store in   = \rightSupport,
}
%
%Definition of the command
\newcommand{\TimberBeam}[1][]{%
  \pgfkeys{/civil/timberBeam, defaultTimberBeam, #1}%
  %Definition of coordinates of the nodes
  \coordinate (p0) at (\beamPosition) {};
  \coordinate (p1) at ([shift={(\beamWidth,0)}]p0) {};
  \coordinate (p2) at ([shift={(\beamWidth,\beamHeight)}]p0) {};
  \coordinate (p3) at ([shift={(0,\beamHeight)}]p0) {};
  %Center of the hole
  \coordinate (p4) at ([shift={(\holePosX,\holePosY)}]p0) {};
  %Draw beam:
  \draw (p0) -- (p1);
  \draw (p1) -- (p2);
  \draw (p2) -- (p3);
  \draw (p3) -- (p0);
  %Supports are drawn
  \ifsupport
    \Support[position = p0, type=\leftSupport];
    \Support[position = p1, type=\rightSupport];
  \fi
%
}

\endinput
